%%=============================================================================
%% Methodologie
%%=============================================================================

\chapter{\IfLanguageName{dutch}{Methodologie}{Methodology}}%
\label{ch:methodologie}

%% In dit hoofstuk geef je een korte toelichting over hoe je te werk bent
%% gegaan. Verdeel je onderzoek in grote fasen, en licht in elke fase toe wat
%% de doelstelling was, welke deliverables daar uit gekomen zijn, en welke
%% onderzoeksmethoden je daarbij toegepast hebt. Verantwoord waarom je
%% op deze manier te werk gegaan bent.
%% 
%% Voorbeelden van zulke fasen zijn: literatuurstudie, opstellen van een
%% requirements-analyse, opstellen long-list (bij vergelijkende studie),
%% selectie van geschikte tools (bij vergelijkende studie, "short-list"),
%% opzetten testopstelling/PoC, uitvoeren testen en verzamelen
%% van resultaten, analyse van resultaten, ...
%%
%% !!!!! LET OP !!!!!
%%
%% Het is uitdrukkelijk NIET de bedoeling dat je het grootste deel van de corpus
%% van je bachelorproef in dit hoofstuk verwerkt! Dit hoofdstuk is eerder een
%% kort overzicht van je plan van aanpak.
%%
%% Maak voor elke fase (behalve het literatuuronderzoek) een NIEUW HOOFDSTUK aan
%% en geef het een gepaste titel.

%%=============================================================================
%% Methodologie
%%=============================================================================

Om de centrale onderzoeksvraag te beantwoorden, wordt een gestructureerde aanpak gevolgd die is onderverdeeld in verschillende opeenvolgende fasen. Dit proces start op 1 maart 2026 en loopt tot de finale conclusie eind mei 2026. In dit hoofdstuk worden de doelstellingen, methoden en deliverables per fase toegelicht.

\section{Fase 1: Literatuurstudie (Maart -- April)}
De eerste fase, die aanvangt op \textbf{01/03/2026}, vormt het theoretische fundament van de bachelorproef. De focus ligt hierbij op:
\begin{itemize}
    \item \textbf{Syntaxisvergelijking:} Het identificeren van de grootste drempels voor junioren bij de overstap van Python naar COBOL.
    \item \textbf{Gamificatie-analyse:} Het bestuderen van het Octalysis-framework en relevante spelmechanismen voor educatieve software.
\end{itemize}
De resultaten van dit onderzoek bepalen de functionele vereisten voor de volgende fase.

\section{Fase 2: Ontwerp en Ontwikkeling van de PoC (April -- Mei)}
Vanaf \textbf{15/04/2026} wordt de theoretische kennis omgezet in een tastbaar product. 
\begin{itemize}
    \item \textbf{Methodiek:} Er wordt gewerkt volgens een iteratief proces waarbij de abstracte logica van COBOL (zoals de \textit{Divisions} en \textit{PIC-clausules}) wordt vertaald naar gamified uitdagingen. 
    \item \textbf{Deliverable:} Een interactieve webapplicatie die dient als Proof-of-Concept (PoC) voor de effectiviteitsmeting.
\end{itemize}

\section{Fase 3: Evaluatie en Resultaten (Mei)}
Na de oplevering van de PoC op \textbf{10/05/2026} volgt de cruciale evaluatiefase. Hierin wordt getoetst of de vooropgestelde doelen worden behaald.
\begin{itemize}
    \item \textbf{Kwantitatief:} Gebruik van de System Usability Scale (SUS) om de bruikbaarheid van de tool te evalueren.
    \item \textbf{Kwalitatief:} Afname van interviews en observaties bij studenten Toegepaste Informatica om de motivatie-impact en leer-efficiëntie te meten in vergelijking met traditionele documentatie.
\end{itemize}

\section{Fase 4: Conclusie (Eind Mei)}
In de finale fase, die start op \textbf{29/05/2026}, worden alle verzamelde data geanalyseerd en gesynthetiseerd. Dit resulteert in een definitief antwoord op de onderzoeksvraag en aanbevelingen voor de verdere modernisering van mainframe-onderwijs.
