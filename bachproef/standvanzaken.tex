\chapter{\IfLanguageName{dutch}{Stand van zaken}{State of the art}}%
\label{ch:stand-van-zaken}

In dit hoofdstuk wordt de theoretische basis gelegd voor het verdere onderzoek. 
Er wordt gestart met een brede verkenning van de mainframe-sector en de huidige demografische verschuivingen, 
waarna er specifiek wordt ingezoomd op de Unisys-architectuur en het DMSII-databasesysteem. 
Tot slot worden de educatieve kaders besproken die als fundament dienen voor het te ontwikkelen competentieframework.

% ----------------------------------------------------------------------------------------------------- %
\section{De Mainframe Wereld: Kritieke Infrastructuur in Transitie}
\label{sec:mainframe-wereld}

\subsection{Wat is een mainframe?}

Om de rol van een Database Administrator (DBA) binnen een mainframe-omgeving te kaderen, is het essentieel om eerst een mainframe te definiëren. 
Volgens de definitie van \textcite{IBM2010} is een mainframe een server dat door bedrijven gebruikt om commerciële databases, transactieservers en applicaties te hosten die een hogere mate van beveiliging en beschikbaarheid vereisen dan doorgaans beschikbaar is op kleinere machines.

Deze servers draaien op gespecialiseerde besturingssystemen zoals z/OS (IBM), ClearPath MCP of OS 2200 (Unisys) en in toenemende mate op specifieke mainframe-distributies van Linux (zoals LinuxONE). 

Deze besturingssystemen, en vaak ook de hardware waarop ze draaien, zijn inherent ontworpen met de strenge eisen van een mainframe-omgeving in gedachten, waarbij een combinatie van de volgende kenmerken centraal staan: \autocite{Ebbers2011}

\begin{itemize}
    \item Compatibiliteit met System z\footnote{System Z is een overkoepelende naam voor een IBM Mainframe.}-besturingssystemen, -toepassingen en -gegevens.
    \item Gecentraliseerde controle van bronnen.
    \item Hardware en besturingssystemen die toegang tot schijfstations kunnen delen met
          andere systemen, met automatische vergrendeling en bescherming tegen destructief
          gelijktijdig gebruik van schijfgegevens.
    \item Een werkwijze waarbij vaak gespecialiseerde medewerkers betrokken zijn die gebruikmaken van
          gedetailleerde procedurehandboeken en zeer georganiseerde procedures voor
          back-ups, herstel, training en noodherstel op een alternatieve locatie.
    \item Hardware en besturingssystemen die routinematig werken met honderden of
          duizenden gelijktijdige I/O-bewerkingen.
    \item Clustertechnologieën waarmee de klant meerdere exemplaren van
          het besturingssysteem als één systeem kan gebruiken. Deze configuratie, bekend als
          Parallel Sysplex, is qua concept vergelijkbaar met een UNIX-cluster, maar maakt het mogelijk
          om systemen naar behoefte toe te voegen of te verwijderen, terwijl applicaties blijven
          draaien. Deze flexibiliteit stelt mainframeklanten in staat om nieuwe applicaties te introduceren
          of het gebruik van bestaande applicaties te staken, in reactie op veranderingen in
          de bedrijfsactiviteiten.
    \item Extra mogelijkheden voor het delen van gegevens en bronnen. In een Parallel Sysplex is het bijvoorbeeld
          mogelijk voor gebruikers van meerdere systemen om tegelijkertijd toegang te krijgen tot dezelfde
          databases, waarbij de toegang tot de database op recordniveau wordt gecontroleerd.
    \item Geoptimaliseerd voor I/O voor bedrijfsgerelateerde gegevensverwerkingstoepassingen
          die ondersteuning bieden voor hogesnelheidsnetwerken en terabytes aan schijfopslag.
\end{itemize}

Hoewel deze kenmerken specifiek door IBM zijn gedefinieerd, maken Unisys-mainframes gebruik van gelijkaardige technologieën en mechanismen.

\subsection{Strategische relevantie in 2026}

\subsection{De "Silver Tsunami" en de generatieshift}

\subsection{De Kenniskloof (Skills Gap) in het IT-onderwijs}

% ----------------------------------------------------------------------------------------------------- %
\section{Het Unisys ClearPath MCP Ecosysteem}
\label{sec:unisys-ecosysteem}

\subsection{De ClearPath Architectuur: E-mode en Stack-design}


\subsection{Het Master Control Program (MCP)}

\subsection{Onderscheid met gedistribueerde systemen}

% ----------------------------------------------------------------------------------------------------- %
\section{Data Management System II (DMSII)}
\label{sec:dmsii}

\subsection{Conceptuele basis: Het hiërarchische en netwerk-datamodel}

\subsection{DASDL: Data and Structure Definition Language}

\subsection{Fysieke organisatie en opslagmanagement}

\subsection{DMSII versus Relationele Databases (SQL)}

% ----------------------------------------------------------------------------------------------------- %
\section{De Rol van de Database Administrator (DBA) Level 1}
\label{sec:rol-dba}

\subsection{Operationele kerntaken en verantwoordelijkheden}

\subsection{De impact van data-integriteit in mainframe-omgevingen}

\subsection{De noodzaak voor formalisering van het leertraject}

% ----------------------------------------------------------------------------------------------------- %
\section{Educatieve Theorieën voor Competentieontwikkeling}
\label{sec:educatieve-theorieen}

\subsection{De Herziene Taxonomie van Bloom}

\subsection{Competentiegericht leren: Kennis, Vaardigheden en Attitude}

\subsection{Microlearning en didactiek voor de nieuwe generatie}

% ----------------------------------------------------------------------------------------------------- %
\section{Bestaande Frameworks en Benchmarking}
\label{sec:bestaande-frameworks}

\subsection{Algemene IT-kaders: e-CF en SFIA}

\subsection{Mainframe-specifieke kaders: Het IBM Z model}

\subsection{De lacune voor het Unisys-platform}

% ----------------------------------------------------------------------------------------------------- %
\section{Conclusie: Richting een Unisys DBA-framework}
\label{sec:conclusie-state-of-the-art}