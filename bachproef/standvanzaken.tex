\chapter{\IfLanguageName{dutch}{Stand van zaken}{State of the art}}%
\label{ch:stand-van-zaken}

In dit hoofdstuk wordt de theoretische basis gelegd voor het verdere onderzoek. 
Er wordt gestart met een brede verkenning van de mainframe-sector en de huidige demografische verschuivingen, 
waarna er specifiek wordt ingezoomd op de Unisys-architectuur en het DMSII-databasesysteem. 
Tot slot worden de educatieve kaders besproken die als fundament dienen voor het te ontwikkelen competentieframework.

% ----------------------------------------------------------------------------------------------------- %
\section{De Mainframe Wereld: Kritieke Infrastructuur in Transitie}
\label{sec:mainframe-wereld}

\subsection{Wat is een mainframe?}

Om de rol van een Database Administrator (DBA) binnen een mainframe-omgeving te kaderen, is het essentieel om eerst een mainframe te definiëren. 
Volgens de definitie van \textcite{IBM2010} is een mainframe een server dat door bedrijven gebruikt om commerciële databases, transactieservers en applicaties te hosten die een hogere mate van beveiliging en beschikbaarheid vereisen dan doorgaans beschikbaar is op kleinere machines.

Deze servers draaien op gespecialiseerde besturingssystemen zoals z/OS (IBM), ClearPath MCP of OS 2200 (Unisys) en in toenemende mate op specifieke mainframe-distributies van Linux (zoals LinuxONE). 

Deze besturingssystemen, en vaak ook de hardware waarop ze draaien, zijn inherent ontworpen met de strenge eisen van een mainframe-omgeving in gedachten. Hoewel de onderstaande lijst niet exhaustief is, staan de volgende kenmerken centraal \autocite{Ebbers2011}:

\begin{itemize}
      \item \textbf{Betrouwbaarheid, beschikbaarheid en onderhoudsvriendelijkheid (RAS\footnote{Reliability, availability, and serviceability}):} \autocite{Ebbers2011}
            \begin{itemize}
                  \item Betrouwbaarheid: De hardwarecomponenten van het systeem beschikken over uitgebreide zelfcontrole- en zelfherstelcapaciteiten. De betrouwbaarheid van de software van het systeem is het resultaat van uitgebreide tests en de mogelijkheid om snel updates uit te voeren voor gedetecteerde problemen.
                  \item Beschikbaarheid: Het systeem kan herstellen van een defect onderdeel zonder dat dit invloed heeft op de rest van het draaiende systeem. 
                  \item Onderhoudsvriendelijkheid: Het systeem kan vaststellen waarom een storing is opgetreden.
            \end{itemize}
      \item \textbf{Beveiliging:} Kritieke gegevens moeten veilig worden
            beheerd en gecontroleerd, en tegelijkertijd beschikbaar worden gesteld aan gebruikers
            die bevoegd zijn om ze in te zien. De mainframecomputer heeft uitgebreide mogelijkheden om
            de gegevens van het bedrijf tegelijkertijd te delen met meerdere gebruikers en toch te beschermen. \autocite{Ebbers2011}
      \item \textbf{Schaalbaarheid:} Het vermogen van de hardware, software of een gedistribueerd
            systeem om goed te blijven functioneren wanneer de omvang of het volume ervan verandert. \autocite{Ebbers2011}
      \item \textbf{Blijvende compatibiliteit:} Het vermogen van een applicatie om in het systeem te werken
            of het vermogen om met andere apparaten of programma's te werken \autocite{Ebbers2011}
\end{itemize}

\subsection{De generatieshift}
Hoewel het mainframe een bewezen technologie is die al decennia lang ondernemingen ondersteunt, 
vindt er momenteel een fundamentele verschuiving plaats in wie dit platform beheert. 
Waar de systemen voorheen het domein waren van de 'Baby Boomer'-generatie, is de langverwachte generatiewissel nu een feit. 
Uit recent onderzoek \autocite{BMC2025} blijkt dat inmiddels 66\% van de mainframe-professionals tot de Millennials of Generatie Z behoort, terwijl het aandeel Boomers en Gen X is gekrompen tot minder dan de helft.
%TODO: ook belangrijk - shift gaat verder dan enkel de demografie, ook de "modernisatie" van de mainframe is in (BMC2025, Kyndryl2025)

\subsection{Hedendaagse relevantie}

\subsection{De Kenniskloof (Skills Gap) in het IT-onderwijs}

% ----------------------------------------------------------------------------------------------------- %
\section{Het Unisys ClearPath MCP Ecosysteem}
\label{sec:unisys-ecosysteem}

\subsection{De ClearPath Architectuur: E-mode en Stack-design}


\subsection{Het Master Control Program (MCP)}

\subsection{Onderscheid met gedistribueerde systemen}

% ----------------------------------------------------------------------------------------------------- %
\section{Data Management System II (DMSII)}
\label{sec:dmsii}

\subsection{Conceptuele basis: Het hiërarchische en netwerk-datamodel}

\subsection{DASDL: Data and Structure Definition Language}

\subsection{Fysieke organisatie en opslagmanagement}

\subsection{DMSII versus Relationele Databases (SQL)}

% ----------------------------------------------------------------------------------------------------- %
\section{De Rol van de Database Administrator (DBA) Level 1}
\label{sec:rol-dba}

\subsection{Operationele kerntaken en verantwoordelijkheden}

\subsection{De impact van data-integriteit in mainframe-omgevingen}

\subsection{De noodzaak voor formalisering van het leertraject}

% ----------------------------------------------------------------------------------------------------- %
\section{Educatieve Theorieën voor Competentieontwikkeling}
\label{sec:educatieve-theorieen}

\subsection{De Herziene Taxonomie van Bloom}

\subsection{Competentiegericht leren: Kennis, Vaardigheden en Attitude}

\subsection{Microlearning en didactiek voor de nieuwe generatie}

% ----------------------------------------------------------------------------------------------------- %
\section{Bestaande Frameworks en Benchmarking}
\label{sec:bestaande-frameworks}

\subsection{Algemene IT-kaders: e-CF en SFIA}

\subsection{Mainframe-specifieke kaders: Het IBM Z model}

\subsection{De lacune voor het Unisys-platform}

% ----------------------------------------------------------------------------------------------------- %
\section{Conclusie: Richting een Unisys DBA-framework}
\label{sec:conclusie-state-of-the-art}