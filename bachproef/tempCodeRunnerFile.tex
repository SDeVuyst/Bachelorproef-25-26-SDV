\section{Gamification}

TODO: fix intro en definitie
\textcite{Deterding2011} definiëren gamification als het gebruik van game-\allowbreak ontwerpelementen in niet-\allowbreak game contexten. 
\textcite{Deterding2011} leggen hun definitie uit en stellen dat gamification meer met "gamen" \textit{(ludus)} dan met "playing" \textit{(paidia)} te maken heeft, er zijn namelijk regels die moeten worden gevolgd.
Gamification  maakt  gebruik  van  elementen  uit  game-ontwerp,  maar  kan  niet  als  volwaardige  games  worden  beschouwd.

\subsection{Game-elementen} %TODO

\subsubsection{Veelvoorkomende Game-elementen: PBL} %TODO
Points (Punten)

Badges (Badges/Trofeeën)

Leaderboards (Ranglijsten)

\subsubsection{Naast PBL} %TODO
Narratief (verhaal), feedback-loops, progressiebalken en avatars.


\subsection{Frameworks}

\subsubsection{MDA-framework}
Het MDA-framework, een acroniem voor \textit{Mechanics}, \textit{Dynamics}, en \textit{Aesthetics}, is een formele methode die tracht de kloof tussen game-ontwerp, game-kritiek en technisch onderzoek te overbruggen \autocite{Hunicke2004}. 
In tegenstelling tot traditionele media wordt de inhoud van een game gedefinieerd door het gedrag van het systeem in interactie met de speler \autocite{Hunicke2004}.
\textcite{Hunicke2004} stellen dat games begrepen moeten worden als systemen die gedrag opbouwen via interactie, waarbij het raamwerk drie causale niveaus onderscheidt:

\begin{itemize}
    \item \textbf{Mechanics} beschrijft de specifieke componenten van de game op het niveau van data-representatie en algoritmen \autocite{Hunicke2004}. 
    Dit omvat de acties, regels en controlemechanismen die de ontwerper aan de speler biedt \autocite{Hunicke2004}.
    \item \textbf{Dynamics} beschrijft het \textit{run-time} gedrag van de mechanieken die inwerken op de input van de speler en elkaars output over een tijdsverloop \autocite{Hunicke2004}.
    \item \textbf{Aesthetics} verwijst naar de gewenste emotionele reacties die bij de speler worden opgeroepen tijdens de interactie met het gamesysteem \autocite{Hunicke2004}.
\end{itemize}

Een fundamenteel aspect van dit raamwerk is het verschil in perspectief tussen de ontwerper en de speler \autocite{Hunicke2004}. 
Waar de ontwerper begint bij de \textit{mechanics} om uiteindelijk een bepaalde \textit{aesthetic} te bereiken, ervaart de speler de game als eerste via de \textit{aesthetics}, die voortvloeien uit de \textit{dynamics} en de onderliggende \textit{mechanics} \autocite{Hunicke2004}. 
\textcite{Hunicke2004} beargumenteren dat deze benadering helpt om "ervaringsgericht" in plaats van enkel "functiegericht" te ontwerpen.

\subsubsection{Octalysis-framework}
%TODO: iets zeggen over de PBL Fallacy
Er is ook een populair framework dat specifiek kijkt naar de psychologische drijfveren achter deze gamedesign-\allowbreak elementen: het \textit{Octalysis-\allowbreak framework}, ontwikkeld door \textcite{Chou2015}. 
Waar de definitie van \textcite{Deterding2011} zich sterk richt op de classificatie van elementen, vertrekt Chou vanuit het principe van \textit{Human-Focused Design}.
Dit houdt in dat het ontwerp niet primair draait om de functionele elementen (zoals badges of leaderboards), maar om de manier waarop deze elementen inspelen op de menselijke motivatie en emoties.
Het framework is opgebouwd rond acht fundamentele motivaties, de \textit{Core Drives}.

\begin{figure}[h]
  \centering
  \includegraphics[width=0.2\textwidth]{octaly