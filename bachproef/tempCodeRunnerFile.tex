\subsubsection{Octalysis-framework}
Een ander populair framework dat specifiek kijkt naar de psychologische drijfveren achter deze gamedesign-\allowbreak elementen: het \textit{Octalysis-\allowbreak framework}, ontwikkeld door \textcite{Chou2015}. 
Waar de definitie van \textcite{Deterding2011} zich sterk richt op de classificatie van elementen, vertrekt Chou vanuit het principe van \textit{Human-Focused Design}.
Dit houdt in dat het ontwerp niet primair draait om de functionele elementen (zoals badges of leaderboards), maar om de manier waarop deze elementen inspelen op de menselijke motivatie en emoties.
Het framework is opgebouwd rond acht fundamentele motivaties, de \textit{Core Drives}.

\begin{figure}[H]
  \centering
  \includegraphics[width=0.8\textwidth]{octalysis.jpg}
  \caption{Octalysis Framework \autocite{Chou2015}}
  \label{fig:octalysis}
\end{figure}