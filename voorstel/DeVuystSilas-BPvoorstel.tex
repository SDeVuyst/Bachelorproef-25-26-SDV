%==============================================================================
% Sjabloon onderzoeksvoorstel bachproef
%==============================================================================
% Gebaseerd op document class `hogent-article'
% zie <https://github.com/HoGentTIN/latex-hogent-article>

% Voor een voorstel in het Engels: voeg de documentclass-optie [english] toe.
% Let op: kan enkel na toestemming van de bachelorproefcoördinator!
\documentclass{hogent-article}

% Invoegen bibliografiebestand
\addbibresource{voorstel.bib}

% Informatie over de opleiding, het vak en soort opdracht
\studyprogramme{Professionele bachelor toegepaste informatica}
\course{Bachelorproef}
\assignmenttype{Onderzoeksvoorstel}
% Voor een voorstel in het Engels, haal de volgende 3 regels uit commentaar
% \studyprogramme{Bachelor of applied information technology}
% \course{Bachelor thesis}
% \assignmenttype{Research proposal}

\academicyear{2025-2026}

\title{De efficiëntie van gamificatie bij het aanleren van COBOL aan junior developers.}

\author{Silas De Vuyst}
\email{silas.devuyst@student.hogent.be}

\supervisor[Co-promotor]{S. Willems (Cerence, \href{mailto:stefaan.willems@telenet.be}{stefaan.willems@telenet.be\textsc{}})}

\projectrepo{https://github.com/SDeVuyst/Bachelorproef-25-26-SDV}

% Binnen welke specialisatierichting uit 3TI situeert dit onderzoek zich?
% Kies uit deze lijst:
%
% - Mobile \& Enterprise development
% - AI \& Data Engineering
% - Functional \& Business Analysis
% - System \& Network Administrator
% - Mainframe Expert
% - Als het onderzoek niet past binnen een van deze domeinen specifieer je deze
%   zelf
%
\specialisation{Mainframe Expert}
\keywords{Gamification, COBOL}

\begin{document}

\begin{abstract}
  Dit onderzoek richt zich op het overbruggen van de groeiende kenniskloof in de IT-sector, 
  waar de kritieke maar verouderde COBOL-infrastructuur bedreigd wordt door de massale uitstroom van ervaren ontwikkelaars en een stagnerende instroom van nieuw talent.
  De centrale probleemstelling is dat de huidige, traditionele didactiek niet aansluit bij de behoeften van junior developers, 
  die gewend zijn aan de interactieve leerpaden van moderne talen zoals Python. 
  Om dit op te lossen, wordt onderzocht hoe gamification aan de hand van het Octalysis-framework de drempel voor COBOL-educatie kan verlagen en de motivatie kan verhogen.
  Er wordt een Proof of Concept ontwikkeld in de vorm van een interactieve website, waarbij de nogal complexe taal wordt vertaald naar levels en spelmechanismen. 
  De effectiviteit van deze methode wordt vervolgens getoetst bij studenten Toegepaste Informatica door gebruik te maken van de System Usability Scale en kwalitatieve interviews, 
  met als verwachting dat een gamified benadering de leer-efficiëntie en motivatie verbetert ten opzichte van de traditionele documentatie-gebaseerde opleidingen.
\end{abstract}

\tableofcontents

% De hoofdtekst van het voorstel zit in een apart bestand, zodat het makkelijk
% kan opgenomen worden in de bijlagen van de bachelorproef zelf.
%---------- Inleiding ---------------------------------------------------------

\section{Inleiding}%
\label{sec:inleiding}

% Waarover zal je bachelorproef gaan? Introduceer het thema en zorg dat volgende zaken zeker duidelijk aanwezig zijn:

% \begin{itemize}
%   \item kaderen thema
%   \item de doelgroep
%   \item de probleemstelling en (centrale) onderzoeksvraag
%   \item de onderzoeksdoelstelling
% \end{itemize}

% Denk er aan: een typische bachelorproef is \textit{toegepast onderzoek}, wat betekent dat je start vanuit een concrete probleemsituatie in bedrijfscontext, een \textbf{casus}. Het is belangrijk om je onderwerp goed af te bakenen: je gaat voor die \textit{ene specifieke probleemsituatie} op zoek naar een goede oplossing, op basis van de huidige kennis in het vakgebied.

% De doelgroep moet ook concreet en duidelijk zijn, dus geen algemene of vaag gedefinieerde groepen zoals \emph{bedrijven}, \emph{developers}, \emph{Vlamingen}, enz. Je richt je in elk geval op it-professionals, een bachelorproef is geen populariserende tekst. Eén specifiek bedrijf (die te maken hebben met een concrete probleemsituatie) is dus beter dan \emph{bedrijven} in het algemeen.

% Formuleer duidelijk de onderzoeksvraag! De begeleiders lezen nog steeds te veel voorstellen waarin we geen onderzoeksvraag terugvinden.

% Schrijf ook iets over de doelstelling. Wat zie je als het concrete eindresultaat van je onderzoek, naast de uitgeschreven scriptie? Is het een proof-of-concept, een rapport met aanbevelingen, \ldots Met welk eindresultaat kan je je bachelorproef als een succes beschouwen?

Aan de hand van deze probleemstelling wordt er onderzoek gedaan naar de vraag: 
"Welke competenties zijn vereist voor een Database Administrator (DBA) Level 1 om doeltreffend te opereren binnen een Unisys-mainframeomgeving, en op welke manier kunnen deze competenties worden geformaliseerd in een eenduidig en direct inzetbaar framework voor het onderwijs?".
Naast de onderwoeksvraag worden een aantal deelvragen geformuleerd om de concrete onderzoeksvraag te verduidelijken. De deelvragen die van pas komen zijn:

\begin{enumerate}
  \item Welke technische competenties zijn noodzakelijk voor een DBA Level 1 binnen een Unisys-mainframeomgeving?
  \item Welke niet-technische (professionele) competenties zijn noodzakelijk voor een DBA Level 1?
  \item Hoe worden DBA-competenties momenteel aangeleerd en geëvalueerd binnen bestaande onderwijs- of trainingsprogramma's?
  \item Welke structuur en opbouw zijn geschikt om DBA Level 1-competenties te vertalen naar een eenduidig framework voor het onderwijs?
  \item Aan welke eisen moet een competentieframework voldoen om direct inzetbaar te zijn binnen een onderwijscontext?
\end{enumerate}

%---------- Stand van zaken ---------------------------------------------------

\section{Literatuurstudie}%
\label{sec:literatuurstudie}

Het opstellen van een competentieframework voor een \emph{Database Administrator Level 1} binnen een Unisys-omgeving vraagt om een kritische bestudering van bestaande competentiemodellen en relevante vakliteratuur, met bijzondere aandacht voor technologie-specifieke vereisten.

Algemene competentiemodellen, waaronder het Skills Framework for the Information Age (SFIA), geven een systematisch overzicht van de kennis en vaardigheden die van IT-professionals worden verwacht.
Hoewel databaseadministratie (DBA) binnen SFIA wordt behandeld, ontbreekt er een specifieke uitwerking voor Unisys-omgevingen. \autocite{SFIA2025} 

\textcite{DEL2024} voerde eerder onderzoek uit naar een competentieframework voor een DBA Level 1, maar dit richtte zich specifiek op een z/OS-omgeving.


% Voor literatuurverwijzingen zijn er twee belangrijke commando's:
% \autocite{KEY} => (Auteur, jaartal) Gebruik dit als de naam van de auteur
%   geen onderdeel is van de zin.
% \textcite{KEY} => Auteur (jaartal)  Gebruik dit als de auteursnaam wel een
%   functie heeft in de zin (bv. ``Uit onderzoek door Doll & Hill (1954) bleek
%   ...'')


%---------- Methodologie ------------------------------------------------------
\section{Methodologie}%
\label{sec:methodologie}

Dit onderzoek wordt uitgevoerd in fases, waarbij elke fase bijdraagt aan de ontwikkeling en validatie van een competentieframework voor een Database Administrator (DBA) Level 1 binnen een Unisys-mainframeomgeving.

\subsection{Fase 1: Literatuurstudie}
In de eerste fase van het onderzoek wordt een uitgebreide literatuurstudie uitgevoerd. Het doel van deze fase is om de huidige stand van zaken in kaart te brengen met betrekking tot bestaande competentieframeworks en hun toepassingsgebied. Hierbij wordt nagegaan welke frameworks reeds bestaan, hoe deze zijn opgebouwd en welke tekortkomingen of beperkingen zij vertonen. Daarnaast worden relevante wetenschappelijke en praktijkgerichte publicaties geanalyseerd die betrekking hebben op de vereiste competenties van een DBA Level 1. Deze analyse vormt de theoretische basis voor de verdere ontwikkeling van het voorgestelde competentieframework.

\subsection{Fase 2: Interviews}
Aanvullend op de literatuurstudie worden in de tweede fase semigestructureerde interviews afgenomen met professionals uit de mainframewereld, met een specifieke focus op experts die actief zijn binnen Unisys-omgevingen. Deze interviews hebben als doel inzicht te verkrijgen in de verwachtingen, verantwoordelijkheden en vereiste vaardigheden van een DBA Level 1 in de praktijk. De resultaten uit deze gesprekken dienen ter verfijning en validatie van de bevindingen uit de literatuurstudie.

\subsection{Fase 3: Structureren van het competentieframework}
Op basis van de resultaten uit de literatuurstudie en de interviews wordt in de derde fase het competentieframework gestructureerd. De geïdentificeerde competenties worden geclusterd, geordend en vertaald naar een coherent en eenduidig geheel dat aansluit bij het instapniveau van een DBA Level 1. Hierbij wordt rekening gehouden met de toepasbaarheid binnen een onderwijscontext.

\subsection{Fase 4: Proof of Concept}
In de vierde fase wordt een proof of concept (PoC) ontwikkeld in de vorm van een website. Deze PoC implementeert het ontwikkelde competentieframework en bevat zowel theoretische uitleg als praktische oefeningen. Het doel van deze fase is om na te gaan in welke mate het framework effectief kan worden vertaald naar een digitaal leerplatform dat bruikbaar is voor onderwijsdoeleinden.

\subsection{Fase 5: Resultaten}
Tijdens de vijfde fase wordt de proof of concept getest door een beperkte groep gebruikers. Op basis van hun ervaringen, feedback en observaties worden de resultaten geanalyseerd. Deze analyse biedt inzicht in de bruikbaarheid, duidelijkheid en effectiviteit van het voorgestelde competentieframework en de bijhorende PoC.

\subsection{Fase 6: Conclusie}
In de laatste fase worden de bevindingen uit het onderzoek samengebracht en wordt een algemene conclusie geformuleerd. Hierbij wordt gereflecteerd op de onderzoeksvraag en wordt nagegaan in welke mate de doelstellingen van het onderzoek zijn bereikt. Daarnaast worden mogelijke aanbevelingen voor toekomstig onderzoek en verdere optimalisatie van het framework besproken.

\begin{figure}[h]
  \centering
  \includegraphics[width=0.2\textwidth]{flowchart.png}
  \caption{Flowchart Methodologie}
  \label{fig:flowchart_methodologie_png}
\end{figure}

%---------- Verwachte resultaten ----------------------------------------------
\section{Verwacht resultaat, conclusie}%
\label{sec:verwachte_resultaten}

% Hier beschrijf je welke resultaten je verwacht. Als je metingen en simulaties uitvoert, kan je hier al mock-ups maken van de grafieken samen met de verwachte conclusies. Benoem zeker al je assen en de onderdelen van de grafiek die je gaat gebruiken. Dit zorgt ervoor dat je concreet weet welk soort data je moet verzamelen en hoe je die moet meten.

% Wat heeft de doelgroep van je onderzoek aan het resultaat? Op welke manier zorgt jouw bachelorproef voor een meerwaarde?

% Hier beschrijf je wat je verwacht uit je onderzoek, met de motivatie waarom. Het is \textbf{niet} erg indien uit je onderzoek andere resultaten en conclusies vloeien dan dat je hier beschrijft: het is dan juist interessant om te onderzoeken waarom jouw hypothesen niet overeenkomen met de resultaten.

Op basis van de uitgevoerde literatuurstudie en de interviews met professionals uit de Unisys-mainframewereld wordt verwacht dat dit onderzoek resulteert in een duidelijk afgebakend en onderbouwd competentieframework voor een Database Administrator (DBA) Level 1. Dit framework zal zowel technische als niet-technische competenties omvatten en deze op een gestructureerde en overzichtelijke manier presenteren, afgestemd op het instapniveau van beginnende DBA's.

Daarnaast wordt verwacht dat het ontwikkelde framework bestaande tekortkomingen in huidige competentiemodellen adresseert, zoals een gebrek aan specifieke focus op mainframe- en Unisys-omgevingen en een beperkte aansluiting bij onderwijsdoelstellingen. Door competenties te koppelen aan concrete leerdoelen en niveaubeschrijvingen, zal het framework direct inzetbaar zijn binnen een onderwijscontext.

De proof of concept, gerealiseerd in de vorm van een educatieve website, zal naar verwachting aantonen dat het competentieframework op een toegankelijke en praktische manier kan worden geïmplementeerd. De combinatie van theoretische uitleg en praktijkgerichte oefeningen moet bijdragen aan een beter begrip van de rol en verantwoordelijkheden van een DBA Level~1. Verwacht wordt dat gebruikers de PoC als duidelijk, bruikbaar en relevant ervaren voor hun leerproces.

Uit de evaluatie van de PoC wordt verwacht dat de feedback van de gebruikers waardevolle inzichten oplevert met betrekking tot de toepasbaarheid en volledigheid van het framework. Deze resultaten zullen toelaten om sterke punten te identificeren, maar ook om eventuele verbeterpunten of hiaten bloot te leggen.

Tot slot wordt verwacht dat dit onderzoek een meerwaarde biedt voor zowel onderwijsinstellingen als studenten door een concreet en praktijkgericht kader aan te reiken voor het aanleren van DBA-competenties binnen een Unisys-mainframeomgeving. De bachelorproef kan bovendien dienen als uitgangspunt voor verdere verfijning van het framework of voor toekomstig onderzoek naar hogere competentieniveaus of andere mainframeplatformen.

\printbibliography[heading=bibintoc]

\end{document}