%---------- Inleiding ---------------------------------------------------------
\section{Inleiding}%
\label{sec:inleiding}

De hedendaagse financiële en overheidssector rust op een fundament van decennia-oude technologie: de programmeertaal COBOL. 
Ondanks de opkomst van moderne talen zoals Python of Go, wordt geschat dat wereldwijd nog steeds honderden miljarden regels COBOL-code in productie zijn. 
Het beheer van deze kritieke infrastructuur komt echter in het gedrang door de 'silver tsunami': ervaren Mainframe-\allowbreak ontwikkelaars gaan massaal met pensioen, 
terwijl de instroom van nieuw talent stagneert. 
Jonge IT-professionals, opgegroeid met moderne talen en interactieve leeromgevingen, ervaren een enorme drempel bij het betreden van het COBOL-ecosysteem.

Het fundamentele probleem is de verouderde didactiek. 
Waar moderne programmeertalen aangeleerd worden via interactieve sandboxes, AI-gestuurde feedback en 'just-in-time' leerpaden, 
is het COBOL-\allowbreak onderwijs vaak blijven steken in de methodieken van de vorige eeuw. 
De leerstof is veelal gebaseerd op PDF's, abstracte theoretische kaders en een gebrek aan visuele of interactieve componenten.

Deze traditionele benadering houdt geen rekening met de cognitieve behoeften van de hedendaagse junior ontwikkelaar, voor wie directe feedback en een gevoel van progressie cruciaal zijn. 
De rigide syntaxis van COBOL wordt hierdoor niet alleen als moeilijk ervaren, maar ook als monotoon. 
Hierdoor ontstaat een motivatiecrisis die de instroom naar kritieke IT-functies belemmert.

Gamification, het strategisch inzetten van spelmechanismen zoals beloningssystemen, leaderboards en levels in een niet-\allowbreak spelcontext kan hiervoor een oplossing bieden.
Door gebruik te maken van een interactieve website, opgesteld aan de hand van het Octalysis-\allowbreak framework kan de student de verschillende aspecten van COBOL leren. 
Deze aspecten, zoals de verschillende \textit{divisions}, worden gradueel aangeleerd volgens levels.
Er wordt verwacht dat de student al een stevige programmeerbasis heeft, aangezien COBOL geen taal is waarmee iemand zijn programmeercarriere start.

Aan de hand van deze probleemstelling wordt er onderzoek gedaan naar de centrale onderzoeksvraag: 
\textit{"Hoe effectief is gamification bij het aanleren van COBOL aan junior developers die gewend zijn aan moderne talen zoals Python?"}\break
Om tot een onderbouwd antwoord te komen, worden de volgende deelvragen behandeld:

\begin{enumerate}
    \item Wat zijn de fundamentele verschillen in syntaxis en logica tussen COBOL en moderne talen (Java/Python)?
    \item Welke gamification-elementen bestaan er en welke sluiten het best aan bij het leren van COBOL?
    \item Welke specifieke barrières ervaren junior developers bij het aanleren van COBOL?
    \item Hoe kan een gamified leeromgeving worden ontworpen die de abstracte logica van COBOL vertaalt naar interactieve uitdagingen?
    \item Hoe verhoudt de leer-efficiëntie van een gamified methode zich tot traditionele documentatie gebaseerde trainingen?
\end{enumerate}

Door spelmechanismen te integreren, wil dit onderzoek aantonen of de drempel voor COBOL-educatie verlaagd kan worden, 
de motivatie verhoogd kan worden en de leer-efficiëntie verbeterd kan worden ten opzichte van de traditionele lesmethoden.


%---------- Stand van zaken ---------------------------------------------------
\section{Literatuurstudie}%
\label{sec:literatuurstudie}

\subsection{Gamification}

\textcite{Deterding2011} definiëren gamification als het gebruik van game-\allowbreak ontwerpelementen in niet-\allowbreak game contexten. 
\textcite{Deterding2011} leggen hun definitie uit en stellen dat gamification meer met "gamen" \textit{(ludus)} dan met "playing" \textit{(paidia)} te maken heeft, er zijn namelijk regels die moeten worden gevolgd.
Gamification  maakt  gebruik  van  elementen  uit  game-ontwerp,  maar  kan  niet  als  volwaardige  games  worden  beschouwd.

Om te verduidelijken welke elementen hieronder vallen, onderscheiden de auteurs vijf niveaus van gamedesign-elementen, geordend van concreet naar abstract:
\begin{itemize} 
    \item \textbf{Game interface design patterns:} Dit zijn de meest directe visuele oplossingen en interactiecomponenten, zoals badges, leaderboards en levels. 
    \item \textbf{Game design patterns en mechanics:} Deze omvatten de terugkerende onderdelen van de gameplay die de interactie structureren, zoals tijdslimieten, beperkte middelen of het werken met beurten. 
    \item \textbf{Game design principes en heuristieken:} Dit zijn evaluatieve richtlijnen om een ontwerp aan te pakken of te analyseren, zoals het bieden van duidelijke doelen of een variëteit aan spelstijlen. 
    \item \textbf{Game modellen:} Dit betreft conceptuele modellen van game-componenten of de game-ervaring, zoals het MDA-model (Mechanics, Dynamics, Aesthetics). 
    \item \textbf{Game design methoden:} Dit zijn de specifieke praktijken en ontwerpprocessen uit de gamedesignwereld, waaronder playtesting en playcentric design.
\end{itemize}

Er is ook een populair framework dat specifiek kijkt naar de psychologische drijfveren achter deze gamedesign-\allowbreak elementen: het \textit{Octalysis-\allowbreak framework}, ontwikkeld door \textcite{Chou2015}. 
Waar de definitie van \textcite{Deterding2011} zich sterk richt op de classificatie van elementen, vertrekt Chou vanuit het principe van \textit{Human-Focused Design}.
Dit houdt in dat het ontwerp niet primair draait om de functionele elementen (zoals badges of leaderboards), maar om de manier waarop deze elementen inspelen op de menselijke motivatie en emoties.
Het framework is opgebouwd rond acht fundamentele motivaties, de \textit{Core Drives}. 
Deze \textit{Core Drives} bevatten 8 verschillende aspecten van de menselijke psychologie zoals \textit{Development \& Accomplishment}, \textit{Ownership}, \textit{Social Influence} etc. \autocite{Chou2015}

Een meta-analyse studie van \textcite{Zhan2022} toont ook aan dat gamification een positieve impact heeft op het onderwijs in programmeren. 
Gamification had met name het grootste effect op de motivatie en academische prestaties van studenten, gevolgd door het denkvermogen.

\subsection{COBOL}
COBOL (Common Business Oriented Language), ontwikkeld aan het eind van de jaren 50, is een van de oudste hogere programmeertalen die nog steeds een cruciale rol speelt in de moderne IT-architectuur. 
De taal staat bekend om zijn betrouwbaarheid, efficiëntie, schaalbaarheid en compatibiliteit. 
Hoewel er tegenwoordig modernere talen bestaan, blijven miljarden regels bestaande code in COBOL in productie vanwege de hoge kosten, complexiteit en risico's die gepaard gaan met migratie naar andere platforms. 
Hierdoor blijft de noodzaak voor COBOL-professionals groot, terwijl deze expertise steeds schaarser wordt door de vergrijzing van het huidige personeelsbestand \autocite{Upadhaya2023}

Ondanks deze afhankelijkheid streven veel organisaties naar modernisering of de overstap naar alternatieve talen zoals Java. 
Deze verschuiving is ook merkbaar in het onderwijs; er is een duidelijk gebrek aan academische cursussen die deze programmeertaal opnemen in hun curriculum \autocite{Upadhaya2023}.


%---------- Methodologie ------------------------------------------------------
\section{Methodologie}%
\label{sec:methodologie}

Dit onderzoek volgt een ontwerpgerichte aanpak, waarbij theoretische inzichten over gamification worden vertaald naar een praktisch leerinstrument voor COBOL. 
Het onderzoek is onderverdeeld in de volgende fasen:

\subsection{Fase 1: Literatuurstudie} 
In de eerste fase wordt een fundamentele analyse uitgevoerd op twee gebieden: 
\begin{itemize} 
    \item \textbf{COBOL-syntax en architectuur:} Een studie naar de unieke structuur van COBOL (zoals de rigide divisies) in vergelijking met moderne, objectgeoriënteerde talen zoals Python. 
    Dit dient om de specifieke drempels voor studenten te identificeren.
    \item \textbf{Gamification-frameworks:} Een analyse van bestaande methodes met een focus op het \textit{Octalysis-framework} van \textcite{Chou2015}. 
    Hierbij worden de acht 'Core Drives' geëvalueerd op hun geschiktheid om intrinsieke en extrinsieke motivatie bij programmeurs te stimuleren.
\end{itemize}

\subsection{Fase 2: Ontwerp van Proof of Concept} 
Op basis van de literatuurstudie wordt een prototype ontwikkeld in de vorm van een interactieve website.
Afhankelijk van de bevindingen uit de literatuurstudie studie zal deze website bestaan uit verschillende levels, die elk een deel van COBOL-structuur zal behandelen.
Zo kan level 1 de globale structuur zoals de \textit{Identification, Environment, Data} en \textit{Procedure Divisions} behandelen, terwijl latere levels ingaan op variabelen, string manipulatie etc.

\subsection{Fase 3: Evaluatie en Resultaten}
Na de realisatie van de PoC wordt de effectiviteit van het instrument getoetst door middel van een gebruikersonderzoek.
De testpopulatie bestaat uit studenten Toegepaste Informatica, onderverdeeld in twee focusgroepen:
\begin{enumerate}
    \item \textbf{De referentiegroep:} Studenten met voorafgaande kennis van COBOL.
    \item \textbf{De doelgroep:} Studenten zonder COBOL-ervaring, maar met kennis van moderne programmeertalen.
\end{enumerate}

De evaluatie vindt plaats via een \textit{mixed-\break methods} aanpak: kwantitatieve data wordt verzameld via de \textit{System Usability Scale (SUS)} voor gebruiksvriendelijkheid, terwijl kwalitatieve inzichten worden verkregen via een semi-gestructureerd interview of survey over de ervaren motivatie.

\subsection{Fase 4: Conclusie}
In de laatste fase worden de bevindingen uit het onderzoek samengebracht en wordt een algemene conclusie geformuleerd. 
Hierbij wordt gereflecteerd op de onderzoeksvraag en wordt nagegaan in welke mate de doelstellingen van het onderzoek zijn bereikt.
Daarnaast worden mogelijke aanbevelingen voor de tool gedaan.

\begin{figure}[h]
  \centering
  \includegraphics[width=0.2\textwidth]{flowchart.png}
  \caption{Tijdlijn Methodologie}
  \label{fig:flowchart_methodologie_png}
\end{figure}


%---------- Verwachte resultaten ----------------------------------------------
\section{Verwacht resultaat, conclusie}%
\label{sec:verwachte_resultaten}

Er wordt verwacht dat de Proof of Concept een significant verschil zal maken in de leerervaring van beide testgroepen:
\begin{itemize}
    \item \textbf{Studenten zonder voorkennis:} De verwachting is dat gamification-elementen de drempelvrees voor de verouderde COBOL-syntax verlaagt.
    \item \textbf{Studenten met voorkennis:} Voor deze groep zal de meerwaarde vermoedelijk liggen in de verhoogde \textit{engagement}. 
    Waar de traditionele syntax als droog wordt ervaren, zorgt het Octalysis-framework voor een meer stimulerende leeromgeving.
\end{itemize}

Om de effectiviteit te meten, zal de \textit{System Usability Scale (SUS)} worden ingezet. 
De hypothese is dat de PoC een SUS-score van minstens 70 zal behalen (bij beide groepen), wat duidt op een "goede" bruikbaarheid.
Bovendien wordt verwacht dat de kwalitatieve interviews zullen uitwijzen dat er een significant verhoogde motivatie is bij de testgebruikers.

De meerwaarde van deze bachelorproef ligt in het overbruggen van de \textit{skills gap} in de IT-sector.
COBOL is nog steeds kritiek voor de financiële wereld, maar de instroom van nieuw talent is laag.
Dit onderzoek toont aan of de moderne leermethodiek gamification ingezet kan worden om legacy-systemen aantrekkelijk te maken voor een nieuwe generatie ontwikkelaars.