%---------- Inleiding ---------------------------------------------------------

\section{Inleiding}
\label{sec:inleiding}

Binnen grote organisaties die bedrijfskritische informatiesystemen beheren, blijven mainframeplatformen een essentiële rol spelen. Ondanks de opkomst van cloud- en gedistribueerde systemen worden Unisys-\allowbreak mainframes nog steeds intensief ingezet in sectoren waar betrouwbaarheid, continuïteit en gegevensintegriteit cruciaal zijn, zoals financiën, overheid en logistiek. Het beheer van databanken binnen deze omgevingen vereist specifieke kennis en vaardigheden die afwijken van hedendaagse relationele of cloudgebaseerde databaseplatformen.

In de praktijk stellen organisaties vast dat het opleiden en inzetten van startende Database Administrators (DBA's) binnen een Unisys-\allowbreak mainframe\allowbreak context een uitdaging vormt. Junior profielen beschikken vaak over een algemene IT- of databankachtergrond, maar missen de specifieke technische en professionele competenties die noodzakelijk zijn om efficiënt en veilig te functioneren in een mainframeomgeving. Bovendien ontbreekt het binnen het onderwijs aan een duidelijk afgebakend en gestructureerd competentiekader dat deze vereisten vertaalt naar concrete leerdoelen en evaluatiecriteria.

Deze bachelorproef vertrekt vanuit een concrete bedrijfscontext waarin de nood bestaat aan een helder en toepasbaar competentieframework voor een DBA Level 1 binnen een Unisys mainframeomgeving. De doelgroep van dit onderzoek bestaat uit IT-professionals en opleidingsverantwoordelijken die betrokken zijn bij het opleiden, begeleiden of evalueren van beginnende DBA's, evenals onderwijsinstellingen die hun curriculum willen afstemmen op reële noden uit het werkveld.

Aan de hand van deze probleemstelling wordt er onderzoek gedaan naar de centrale onderzoeksvraag: 
\textit{"Welke competenties zijn vereist voor een Database Administrator (DBA) Level 1 om doeltreffend te opereren binnen een Unisys mainframeomgeving, en op welke manier kunnen deze competenties worden geformaliseerd in een eenduidig en direct inzetbaar framework voor het onderwijs?"}\break
Om tot een onderbouwd antwoord te komen, worden de volgende deelvragen behandeld:

\begin{enumerate}
    \item Welke technische competenties zijn noodzakelijk voor een DBA Level 1 binnen een Unisys mainframeomgeving?
    \item Welke niet-\allowbreak technische (professionele) competenties zijn noodzakelijk voor een DBA Level 1?
    \item Hoe worden DBA-\allowbreak competenties momenteel aangeleerd en geëvalueerd binnen bestaande onderwijs- of trainingsprogramma's?
    \item Welke structuur en opbouw zijn geschikt om DBA Level 1-competenties te vertalen naar een eenduidig framework voor het onderwijs?
    \item Aan welke eisen moet een competentieframework voldoen om direct inzetbaar te zijn binnen een onderwijscontext?
\end{enumerate}
De uiteindelijke onderzoeksdoelstelling van deze bachelorproef is het ontwikkelen van een tastbaar hulpmiddel voor het werkveld en het onderwijs. Het concrete eindresultaat is een \textbf{proof-of-concept in de vorm van een interactieve website}. Deze website dient als direct inzetbaar platform waar het geformuleerde competentie-framework wordt gepresenteerd, zodat zowel studenten als docenten en mentoren over een eenduidig instrument beschikken voor de opleiding en evaluatie van junior DBA's.

%---------- Stand van zaken ---------------------------------------------------

\section{Literatuurstudie}%
\label{sec:literatuurstudie}

Het opstellen van een competentieframework voor een \emph{Database Administrator Level 1} binnen een Unisys-\allowbreak omgeving vraagt om een kritische bestudering van bestaande competentiemodellen en relevante vakliteratuur, met bijzondere aandacht voor technologie-\allowbreak specifieke vereisten.

Algemene competentiemodellen, waaronder het Skills Framework for the Information Age (SFIA), geven een systematisch overzicht van de kennis en vaardigheden die van IT-\allowbreak professionals worden verwacht.
Hoewel databaseadministratie (DBA) binnen SFIA wordt behandeld, ontbreekt er een specifieke uitwerking voor Unisys-\allowbreak omgevingen. \autocite{SFIA2025} 

\textcite{DEL2024} voerde eerder onderzoek uit naar een competentieframework voor een DBA Level 1, maar dit richtte zich specifiek op een \mbox{z/OS}-\allowbreak omgeving. Hoewel de fundamentele verantwoordelijkheden van een DBA, zoals het waarborgen van databeschikbaarheid en -beveiliging universeel zijn, verschillen de technische implementatie en de vereiste systeemkennis in een Unisys-\allowbreak omgeving aanzienlijk. Waar \mbox{z/OS} gebruikmaakt van specifieke subsystemen, is een Unisys-\allowbreak omgeving vaak gebaseerd op de ClearPath architectuur, waarbij databanken zoals MCP Enterprise Database Server (ook wel bekend als DMSII) \autocite{Unisys2022} een centrale rol spelen.

Om dit framework bruikbaar te maken voor het onderwijs, zal dit onderzoek tevens gebruikmaken van de herziene Taxonomie van Bloom. Dit stelt ons in staat om per competentie het gewenste cognitieve niveau vast te leggen. Hierbij wordt gefocust op niveaus zoals onthouden, begrijpen en toepassen, wat essentieel is voor een eenduidige profilering van een Level 1-DBA \autocite{Anderson2001}.

% Voor literatuurverwijzingen zijn er twee belangrijke commando's:
% \autocite{KEY} => (Auteur, jaartal) Gebruik dit als de naam van de auteur
%   geen onderdeel is van de zin.
% \textcite{KEY} => Auteur (jaartal)  Gebruik dit als de auteursnaam wel een
%   functie heeft in de zin (bv. ``Uit onderzoek door Doll & Hill (1954) bleek
%   ...'')

%---------- Methodologie ------------------------------------------------------
\section{Methodologie}%
\label{sec:methodologie}

Dit onderzoek wordt uitgevoerd in fases, waarbij elke fase bijdraagt aan de ontwikkeling en validatie van een competentieframework voor een Database Administrator (DBA) Level 1 binnen een Unisys mainframeomgeving.

\subsection{Fase 1: Literatuurstudie}
In de eerste fase van het onderzoek wordt een uitgebreide literatuurstudie uitgevoerd. Het doel van deze fase is om de huidige stand van zaken in kaart te brengen met betrekking tot bestaande competentieframeworks en hun toepassingsgebied. Hierbij wordt nagegaan welke frameworks reeds bestaan, hoe deze zijn opgebouwd en welke tekortkomingen of beperkingen zij vertonen. Daarnaast worden relevante publicaties geanalyseerd die betrekking hebben op de vereiste competenties van een DBA Level 1. Deze analyse vormt de theoretische basis voor de verdere ontwikkeling van het voorgestelde competentieframework.

\subsection{Fase 2: Interviews}
Aanvullend op de literatuurstudie worden in de tweede fase semigestructureerde interviews afgenomen met professionals uit de mainframewereld, met een specifieke focus op experts die actief zijn binnen Unisys-\allowbreak omgevingen. Deze interviews hebben als doel inzicht te verkrijgen in de verwachtingen, verantwoordelijkheden en vereiste vaardigheden van een DBA Level 1 in de praktijk. De resultaten uit deze gesprekken dienen ter verfijning en validatie van de bevindingen uit de literatuurstudie.

\subsection{Fase 3: Structureren van het competentieframework}
Op basis van de resultaten uit de literatuurstudie en de interviews wordt in de derde fase het competentieframework gestructureerd. Hierbij wordt de herziene Taxonomie van Bloom \autocite{Anderson2001} gehanteerd om de competenties te koppelen aan specifieke cognitieve niveaus. Dit waarborgt dat de leerdoelen eenduidig zijn geformuleerd en aansluiten bij het instapniveau van een DBA Level 1 binnen een onderwijscontext.

\subsection{Fase 4: Proof of Concept}
In de vierde fase wordt een proof of concept (PoC) ontwikkeld in de vorm van een website. Deze PoC implementeert het ontwikkelde competentieframework en bevat zowel theoretische uitleg als praktische oefeningen. Het doel van deze fase is om na te gaan in welke mate het framework effectief kan worden vertaald naar een digitaal leerplatform dat bruikbaar is voor onderwijsdoeleinden.

\subsection{Fase 5: Resultaten}
Tijdens de vijfde fase wordt de proof of concept getest door een beperkte groep gebruikers. Op basis van hun ervaringen, feedback en observaties worden de resultaten geanalyseerd. Deze analyse biedt inzicht in de bruikbaarheid, duidelijkheid en effectiviteit van het voorgestelde competentieframework en de bijhorende PoC.

\subsection{Fase 6: Conclusie}
In de laatste fase worden de bevindingen uit het onderzoek samengebracht en wordt een algemene conclusie geformuleerd. Hierbij wordt gereflecteerd op de onderzoeksvraag en wordt nagegaan in welke mate de doelstellingen van het onderzoek zijn bereikt. Daarnaast worden mogelijke aanbevelingen voor toekomstig onderzoek en verdere optimalisatie van het framework besproken.

\begin{figure}[h]
  \centering
  \includegraphics[width=0.2\textwidth]{flowchart.png}
  \caption{Flowchart Methodologie}
  \label{fig:flowchart_methodologie_png}
\end{figure}

%---------- Verwachte resultaten ----------------------------------------------
\section{Verwacht resultaat en meerwaarde}
\label{sec:verwachte_resultaten}

% Hier beschrijf je welke resultaten je verwacht. Als je metingen en simulaties uitvoert, kan je hier al mock-ups maken van de grafieken samen met de verwachte conclusies. Benoem zeker al je assen en de onderdelen van de grafiek die je gaat gebruiken. Dit zorgt ervoor dat je concreet weet welk soort data je moet verzamelen en hoe je die moet meten.

% Wat heeft de doelgroep van je onderzoek aan het resultaat? Op welke manier zorgt jouw bachelorproef voor een meerwaarde?

% Hier beschrijf je wat je verwacht uit je onderzoek, met de motivatie waarom. Het is \textbf{niet} erg indien uit je onderzoek andere resultaten en conclusies vloeien dan dat je hier beschrijft: het is dan juist interessant om te onderzoeken waarom jouw hypothesen niet overeenkomen met de resultaten.

Op basis van de uitgevoerde literatuurstudie en de interviews met professionals uit de Unisys-\allowbreak mainframewereld wordt verwacht dat dit onderzoek resulteert in een onderbouwd competentieframework voor een DBA Level 1. Dit framework zal technische en niet-technische competenties structureren volgens de herziene Taxonomie van Bloom \autocite{Anderson2001}, waardoor leerdoelen specifiek worden afgestemd op de cognitieve niveaus 'onthouden', 'begrijpen' en 'toepassen'.

Daarnaast wordt verwacht dat het framework de kloof tussen algemene IT-opleidingen en de specifieke Unisys-niche dicht. De proof of concept (PoC), een educatieve website, zal de praktische inzetbaarheid van dit framework aantonen door theoretische modules te combineren met interactieve oefeningen.

Om de effectiviteit van de PoC te meten, wordt een kwantitatieve evaluatie uitgevoerd bij de doelgroep. Hierbij wordt data verzameld aan de hand van een 5-punts Likert-schaal. De resultaten zullen worden gevisualiseerd in een staafdiagram, waarbij de X-as de criteria \textbf{gebruiksvriendelijkheid}, \textbf{relevantie van de leerstof} en \textbf{duidelijkheid van de oefeningen} bevat, en de Y-as de gemiddelde score weergeeft. De verwachting is dat alle onderdelen een score van minstens 4.0 behalen.

Uit de kwalitatieve feedback wordt verwacht dat professionals de volledigheid van het framework bevestigen, terwijl studenten de website zullen waarderen als een drempelverlagend instrument voor een complex platform. De bachelorproef biedt hiermee een meerwaarde voor zowel het onderwijs als de industrie door een gestandaardiseerd kader te bieden voor de opleiding van junior mainframe-\allowbreak specialisten.