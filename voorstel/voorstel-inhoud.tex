%---------- Inleiding ---------------------------------------------------------
\section{Inleiding}%
\label{sec:inleiding}

De hedendaagse financiële en overheidssector rust op een fundament van decennia-oude technologie: de programmeertaal COBOL. 
Ondanks de opkomst van moderne talen zoals Python of Go, wordt geschat dat wereldwijd nog steeds honderden miljarden regels COBOL-code in productie zijn. 
Het beheer van deze kritieke infrastructuur komt echter in het gedrang door de 'silver tsunami': ervaren Mainframe-ontwikkelaars gaan massaal met pensioen, terwijl de instroom van nieuw talent stagneert. 
Jonge IT-professionals, opgegroeid met moderne talen en interactieve leeromgevingen, ervaren een enorme drempel bij het betreden van het COBOL-ecosysteem.

Het fundamentele probleem is de verouderde didactiek. Waar moderne programmeertalen aangeleerd worden via interactieve sandboxes, AI-gestuurde feedback en 'just-in-time' leerpaden, 
is het COBOL-onderwijs vaak blijven steken in de methodieken van de vorige eeuw. 
De leerstof is veelal gebaseerd op lijvige, statische PDF's, abstracte theoretische kaders en een gebrek aan visuele of interactieve componenten.

Deze traditionele benadering houdt geen rekening met de cognitieve behoeften van de hedendaagse junior ontwikkelaar, voor wie directe feedback en een gevoel van progressie cruciaal zijn. 
De rigide syntaxis van COBOL wordt hierdoor niet alleen als moeilijk ervaren, maar ook als monotoon. 
Hierdoor ontstaat een motivatiecrisis die de instroom naar kritieke IT-functies belemmert.

Gamification, het strategisch inzetten van spelmechanismen zoals beloningssystemen, narratieve elementen en feedback-loops in een niet-spelcontext kan hiervoor een oplossing bieden.
Door gebruik te maken van een interactieve website, opgesteld aan de hand van het Octalysis-framework kan de student de verschillende aspecten van COBOL leren. 
Deze aspecten, zoals de verschillende \textit{divisions}, worden gradueel aangeleerd volgens levels.
Er wordt verwacht dat de student al een stevige programmeerbasis heeft, aangezien COBOL geen taal is waarmee iemand zijn programmeercarriere start.


Aan de hand van deze probleemstelling wordt er onderzoek gedaan naar de centrale onderzoeksvraag: 
\textit{"Hoe effectief is gamification bij het aanleren van COBOL aan junior developers die gewend zijn aan moderne talen zoals Python of Java?"}\break
Om tot een onderbouwd antwoord te komen, worden de volgende deelvragen behandeld:

\begin{enumerate}
    \item Wat zijn de fundamentele verschillen in syntaxis en logica tussen COBOL en moderne talen (Java/Python)?
    \item Welke gamification-elementen bestaan er en welke sluiten het best aan bij het leren van COBOL?
    \item Welke specifieke barrières ervaren junior developers bij het aanleren van COBOL?
    \item Hoe kan een gamified leeromgeving worden ontworpen die de abstracte logica van COBOL vertaalt naar interactieve uitdagingen?
    \item Hoe verhoudt de leer-efficiëntie van een gamified methode zich tot traditionele documentatie-gebaseerde trainingen?
\end{enumerate}

Door spelmechanismen te integreren, wil dit onderzoek aantonen of de drempel voor COBOL-educatie verlaagd kan worden, de motivatie verhoogd kan worden en de leer-efficiëntie verbeterd kan worden ten opzichte van traditionele, statische lesmethoden.


%---------- Stand van zaken ---------------------------------------------------
\section{Literatuurstudie}%
\label{sec:literatuurstudie}

\subsection{Gamification}

Volgens \textcite{Deterding2011} is gamificatie het gebruik van game-ontwerpelementen in niet-game-contexten. 
\textcite{Deterding2011} leggen hun definitie uit en stellen dat gamification meer met "gamen" \textit{(ludus)} dan met "playing" \textit{(paidia)} te maken heeft, er zijn namelijk regels die moeten worden gevolgd.
Gamification  maakt  gebruik  van  elementen  uit  game-ontwerp,  maar  kan  niet  als  volwaardige  games  worden  beschouwd.

Om te verduidelijken welke elementen hieronder vallen, onderscheiden de auteurs vijf niveaus van gamedesign-elementen, geordend van concreet naar abstract:
\begin{itemize} 
    \item \textbf{Game interface design patterns:} Dit zijn de meest directe visuele oplossingen en interactiecomponenten, zoals badges, leaderboards en levels. 
    \item \textbf{Game design patterns en mechanics:} Deze omvatten de terugkerende onderdelen van de gameplay die de interactie structureren, zoals tijdslimieten, beperkte middelen of het werken met beurten. 
    \item \textbf{Game design principes en heuristieken:} Dit zijn evaluatieve richtlijnen om een ontwerp aan te pakken of te analyseren, zoals het bieden van duidelijke doelen of een variëteit aan spelstijlen. 
    \item \textbf{Game modellen:} Dit betreft conceptuele modellen van game-componenten of de game-ervaring, zoals het MDA-model (Mechanics, Dynamics, Aesthetics). 
    \item \textbf{Game design methoden:} Dit zijn de specifieke praktijken en ontwerpprocessen uit de gamedesignwereld, waaronder playtesting en playcentric design.
\end{itemize}

Een meta-analyse studie van \textcite{Zhan2022} toont ook aan dat gamificatie een positieve impact heeft op het onderwijs in programmeren. 
Gamificatie had met name het grootste effect op de motivatie en academische prestaties van studenten, gevolgd door het denkvermogen.


\subsection{COBOL}
COBOL (Common Business Oriented Language), ontwikkeld aan het eind van de jaren 50, is een van de oudste hogere programmeertalen die nog steeds een cruciale rol speelt in de moderne IT-architectuur. 
De taal staat bekend om zijn betrouwbaarheid, efficiëntie, schaalbaarheid en compatibiliteit. 
Hoewel er tegenwoordig modernere talen bestaan, blijven miljarden regels bestaande code in COBOL in productie vanwege de hoge kosten, complexiteit en risico's die gepaard gaan met migratie naar andere platforms. 
Hierdoor blijft de noodzaak voor COBOL-professionals groot, terwijl deze expertise steeds schaarser wordt door de vergrijzing van het huidige personeelsbestand \autocite{Upadhaya2023}

Ondanks deze afhankelijkheid streven veel organisaties naar modernisering of de overstap naar alternatieve talen zoals Java. 
Deze verschuiving is ook merkbaar in het onderwijs; er is een duidelijk gebrek aan academische cursussen die deze programmeertaal opnemen in hun curriculum \autocite{Upadhaya2023}.


%---------- Methodologie ------------------------------------------------------
\section{Methodologie}%
\label{sec:methodologie}

Hier beschrijf je hoe je van plan bent het onderzoek te voeren. Welke onderzoekstechniek ga je toepassen om elk van je onderzoeksvragen te beantwoorden? Gebruik je hiervoor literatuurstudie, interviews met belanghebbenden (bv.~voor requirements-analyse), experimenten, simulaties, vergelijkende studie, risico-analyse, PoC, \ldots?

Valt je onderwerp onder één van de typische soorten bachelorproeven die besproken zijn in de lessen Research Methods (bv.\ vergelijkende studie of risico-analyse)? Zorg er dan ook voor dat we duidelijk de verschillende stappen terug vinden die we verwachten in dit soort onderzoek!

Vermijd onderzoekstechnieken die geen objectieve, meetbare resultaten kunnen opleveren. Enquêtes, bijvoorbeeld, zijn voor een bachelorproef informatica meestal \textbf{niet geschikt}. De antwoorden zijn eerder meningen dan feiten en in de praktijk blijkt het ook bijzonder moeilijk om voldoende respondenten te vinden. Studenten die een enquête willen voeren, hebben meestal ook geen goede definitie van de populatie, waardoor ook niet kan aangetoond worden dat eventuele resultaten representatief zijn.

Uit dit onderdeel moet duidelijk naar voor komen dat je bachelorproef ook technisch voldoen\-de diepgang zal bevatten. Het zou niet kloppen als een bachelorproef informatica ook door bv.\ een student marketing zou kunnen uitgevoerd worden.

Je beschrijft ook al welke tools (hardware, software, diensten, \ldots) je denkt hiervoor te gebruiken of te ontwikkelen.

Probeer ook een tijdschatting te maken. Hoe lang zal je met elke fase van je onderzoek bezig zijn en wat zijn de concrete \emph{deliverables} in elke fase?

%---------- Verwachte resultaten ----------------------------------------------
\section{Verwacht resultaat, conclusie}%
\label{sec:verwachte_resultaten}

Hier beschrijf je welke resultaten je verwacht. Als je metingen en simulaties uitvoert, kan je hier al mock-ups maken van de grafieken samen met de verwachte conclusies. Benoem zeker al je assen en de onderdelen van de grafiek die je gaat gebruiken. Dit zorgt ervoor dat je concreet weet welk soort data je moet verzamelen en hoe je die moet meten.

Wat heeft de doelgroep van je onderzoek aan het resultaat? Op welke manier zorgt jouw bachelorproef voor een meerwaarde?

Hier beschrijf je wat je verwacht uit je onderzoek, met de motivatie waarom. Het is \textbf{niet} erg indien uit je onderzoek andere resultaten en conclusies vloeien dan dat je hier beschrijft: het is dan juist interessant om te onderzoeken waarom jouw hypothesen niet overeenkomen met de resultaten.
